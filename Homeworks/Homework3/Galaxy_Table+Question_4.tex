\documentclass{article}
\usepackage{graphicx}
\usepackage{booktabs}
\usepackage{rotating}
\usepackage{geometry}
\usepackage{pdfpages}

\begin{document}
\pagestyle{empty}

\begin{sidewaystable}[ht]
\centering
\small

\begin{tabular}{lccccc}
\hline \hline
Galaxy Name & Halo Mass [$10^{12} M_{\odot}$] & Disk Mass [$10^{12} M_{\odot}$] & Bulge Mass [$10^{12} M_{\odot}$] & Total Mass [$10^{12} M_{\odot}$] & Baryon Fraction \\
\hline
Milky Way & 1.975 & 0.075 & 0.010 & 2.060 & 0.0413 \\
Andromeda (M31) & 1.921 & 0.120 & 0.019 & 2.060 & 0.0675 \\
Triangulum (M33) & 0.187 & 0.009 & 0.000 & 0.196 & 0.0459 \\
\hline Local Group & 4.083 & 0.204 & 0.029 & 4.316 & 0.0540 \\
\hline
\end{tabular}
\caption{Shows mass and baryon fraction of galaxy components of the Milky Way, M31, M33, and the local group.}
\label{Table:1}
\end{sidewaystable}


\begin{center}
    {\Large \textbf{Question 4 Answers}}
\end{center}

1. Both the Milky Way and M31 have identical total mass values of approximately 2.060 * $10^{12} M_{\odot}$. In both cases the total mass is dominated by the dark matter halo; however, the Milky Way has a more massive dark matter halo compared to M31.\vspace{5mm}

2. The stellar mass of both the disk and bulge of M31 is far larger than that of the Milky Way totaling 0.085 * $10^{12} M_{\odot}$ compared to 0.139 * $10^{12} M_{\odot}$. Thus, the luminosity of M31 is greater than the Milky Way as stellar mass is directly related to luminosity. Essentially, a greater stellar mass indicates more mass is available to create stars thus increasing luminosity.\vspace{5mm}

3. Conversely the total dark matter mass of the Milky Way is greater than M31, approximately 1.028 times greater. This is somewhat surprising as if we assume the baryon ratio of galaxies to be a constant, we would expect galaxies of similar mass to have similar dark matter and stellar halo masses. However, we instead find that the milky way has a smaller baryon ratio compared to M31.\vspace{5mm}

4. The baryon ratio of each galaxy is given in Table 1. As can be clearly seen, the baryon fraction of the three major objects in the Local Group is significantly smaller than the ratio in the universe as a whole of approximately \%16. This indicates that there is far more dark matter or far less baryonic matter located within these galaxies compared to the universe as a whole. Possible explanations for this include that dark matter conglomerates more readily than baryonic matter leading to higher concentrations in galaxies, or there is a significant amount of baryonic matter located outside of galaxies, potentially in galactic filaments or diffuse gas comprising the circumgalactic medium. 

\end{document}